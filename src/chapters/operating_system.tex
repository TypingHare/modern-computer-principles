\chapter{操作系统}\label{ch:os}

在计算机领域中,一切看得见摸得着的实物称为\textbf{硬件}。在上一章里我们已经粗略地学习了重要的硬件。然而,光有硬件是不够的,我们还要让这些硬件有机地结合在一起,发挥出特定的作用。指导硬件工作的抽象实体称为\textbf{软件}。虽然软件本身看不见摸不着,但软件借助硬件作为媒介使人们更加轻松地与计算机硬件进行交互,并实现特定的功能。

软件分为\textbf{操作系统}和\textbf{应用程序}。本章我们先介绍\textbf{操作系统}。它是直接与硬件交互的特殊软件。若将计算机软件比作楼房,操作系统毫无疑问就是地基;应用程序则是拔地而起的高楼中的一个个隔间。


\section{应用程序与进程}\label{sec:os_app_process}

上一章中我们粗略地讨论了外存和内存之间的关系,着重强调了任何应用程序都要被复制到内存中,才能被CPU执行。问题是:到底应用程序是什么?它又是如何被复制到内存中的呢?

比起直接写出定义,我们先来说文解字一下。“程”字有“事情发展的步骤”的含义,常见的组词有“日程”和“课程”。“序”表示一种定向的排列,即“顺序”。我们看一个现实中的例子。以下是小明同学的一天:

\begin{itemize}
    \item { 07:00 起床,洗脸刷牙 }
    \item { 07:30 搭车回学校 }
    \item { 12:00 去饭堂吃饭 }
    \item { 16:30 放学回家 }
    \item { 18:00 吃晚饭 }
    \item { 22:00 洗澡睡觉 }
\end{itemize}

上述的列表是小明的日程。小明按顺序一件件地完成当日的任务。如果每一个任务称作“事件”,那么我们称这个日程为“有序事件集”。

程序,本质上就是一系列的\textbf{指令}。CPU在执行一个程序时,会从程序的第一个指令开始执行,直到遇到“结束”指令。指令的种类有很多,比如对两个数进行加法,读取内存中的一个数,和将数据写入到内存中。不同的指令被表示为不同的比特流。比如说,加法指令可能被表示为$0011$,读取指令可能被表示为$0101$,写入指令可能表示为$1110$。这些指令所对应的比特流是CPU的厂商制定的。故而同样的指令,在不同的CPU上对应的比特流也不同。我们用的一切软件,本质上都是一个个的程序。小到“小而美”的微信,大到臃肿不堪的Windows11操作系统,都是这样的一组“有序指令集”。

如果说“指令”是一个比特流,那么作为“有序指令集”的程序也可以表示为一个比特流。这个比特流被操作系统存放在外存中。当我们需要使用这个程序时,操作系统再将这个程序复制一份到内存中。当一个程序被加载到内存中后,它就被称为“进程”。在内存中可以同时存在多个对应着同一个应用程序的进程。

你可能会感到疑惑:如果操作系统负责将应用程序加载到内存里,那么作为软件的操作系统又是如何被加载到内存中的呢?实际上,计算机中有一个“引导程序”,其被刻在主板一个部件中。启动计算机时,引导程序会被执行,将操作系统加载到内存中。这一过程俗称为\textbf{开机}。

\section{内存管理}\label{sec:os_memory_management}

“我手机内存爆了,卡得没法用了!”在手机性能还比较羸弱的年代,这是常能听见的话。到底什么是内存?内存爆了是指内存爆炸了吗?我们一起来看看。



\section{文件系统}\label{sec:os:file_system}


\section{输入输出管理}\label{sec:os:io_management}